\documentclass[letterpaper]{article}
\usepackage{amsmath}
\usepackage{tikz}
\usepackage{epigraph}
\usepackage{lipsum}
\usepackage{hyperref}

\usepackage{setspace, amsmath}

\usepackage[centering,includeheadfoot,margin=2cm]{geometry}
\usepackage{xcolor}
\usepackage{calc,blindtext}

\renewcommand\epigraphflush{flushright}
\renewcommand\epigraphsize{\normalsize}
\setlength\epigraphwidth{0.7\textwidth}

\definecolor{titlepagecolor}{cmyk}{1,.60,0,.40}

\DeclareFixedFont{\titlefont}{T1}{ppl}{b}{it}{1.0in}

\makeatletter
\def\printauthor{%
    {\large \@author}}
\makeatother
\author{%
    Nico Taljaard \\
    10153285 \vspace{20pt} \\
    Gerhard Smit \\
    12282945 \vspace{20pt} \\
    Martin Schoeman \\
    10651994 \\
}

% The following code is borrowed from: http://tex.stackexchange.com/a/86310/10898

\newcommand\titlepagedecoration{%
	\begin{tikzpicture}[remember picture,overlay,shorten >= -10pt]
	
		\coordinate (aux1) at ([yshift=-15pt]current page.north east);
		\coordinate (aux2) at ([yshift=-410pt]current page.north east);
		\coordinate (aux3) at ([xshift=-4.5cm]current page.north east);
		\coordinate (aux4) at ([yshift=-150pt]current page.north east);
		
		\begin{scope}[titlepagecolor!40,line width=12pt,rounded corners=12pt]
			\draw
			  (aux1) -- coordinate (a)
			  ++(225:5) --
			  ++(-45:5.1) coordinate (b);
			\draw[shorten <= -10pt]
			  (aux3) --
			  (a) --
			  (aux1);
			\draw[opacity=0.6,titlepagecolor,shorten <= -10pt]
			  (b) --
			  ++(225:2.2) --
			  ++(-45:2.2);
		\end{scope}
			\draw[titlepagecolor,line width=8pt,rounded corners=8pt,shorten <= -10pt]
			  (aux4) --
			  ++(225:0.8) --
			  ++(-45:0.8);
		\begin{scope}[titlepagecolor!70,line width=6pt,rounded corners=8pt]
			\draw[shorten <= -10pt]
			  (aux2) --
			  ++(225:3) coordinate[pos=0.45] (c) --
			  ++(-45:3.1);
			\draw
			  (aux2) --
			  (c) --
			  ++(135:2.5) --
			  ++(45:2.5) --
			  ++(-45:2.5) coordinate[pos=0.3] (d);   
			\draw 
			  (d) -- +(45:1);
		\end{scope}
	\end{tikzpicture}
}

\begin{document}

\begin{titlepage}

\noindent
\titlefont Laminin \par
\epigraph{ Tender application for Community Project for Facial Recognition presented by Quant Solutions. The purpose of this document is to give the client reason to hire us to solve this problem and to show how we will achieve success and create the best possible solution.}%
{\textit{ 2014 }\\ \textsc{ }}
\null\vfill
\vspace*{3cm}
\noindent
\hfill
\begin{minipage}{0.35\linewidth}
    \begin{flushright}
        \printauthor
    \end{flushright}
\end{minipage}
%
\begin{minipage}{0.02\linewidth}
    \rule{1pt}{125pt}
\end{minipage}
\titlepagedecoration
\end{titlepage}

% % % % % % % % % % % % % % %
% 							%
%	Remainder of document	%
% 							%
% % % % % % % % % % % % % % % 

	\newpage
		\renewcommand\contentsname{TABLE OF CONTENTS}
		\newcommand\contentsnameLC{\colorbox{blue}{\makebox[\textwidth-2\fboxsep][l]{\bfseries\color{white} Table of Contents}}}
		
		\hypersetup{linktocpage}
		\tableofcontents
		
		\newpage
		\section*{\colorbox{blue}{\makebox[\textwidth-2\fboxsep][l]{\bfseries\color{white} Vision \& Objective}}} \addcontentsline{toc}{section}{Vision \& Objective}
		
		\vspace{0.2in}
The vision of this project is to create a simple and effective solution that will allow the users, of the system, to identify any criminal or strange individuals in their neighbourhood, also to aid the police in investigations of criminals in an area by looking at the database of faces captured. 
		\\
		\\
The problem that is to be solved; faces of the people in the community need to be found and stored on the database from different sources, such as different resolutions and quality videos (real time streams) and each face needs to be classified into unique individuals. 
\\
Members of the neighbourhood watch will have access to a web or mobile application where a picture of a suspect can be uploaded and matched with other individuals on the system to see if the criminal is someone in the community. This system will allow verification of locals and reqular visitors from the faces of the intruders or new visitors.
		\\
		\\
The objective of this project is to use technology that is simple to integrate and incorporate, such as the API given to us from the android SDK to have access to the camera of the mobile device, and to store the images recorded on a database using PostgreSQL, also to store all the log data on the server.


		
		\vspace{0.2in}
				
		\section*{\colorbox{blue}{\makebox[\textwidth-2\fboxsep][l]{\bfseries\color{white} Requirements}}} \addcontentsline{toc}{section}{Requirements}
		
		\vspace{0.2in}
		
		\begin{itemize}
			\item The Architectural requirements of the system will be:
			\begin{itemize}
				\item	Video streaming:
				\begin{itemize}				
				\item	Must process real time videos.
				\end{itemize}
				
				\item	Scalability of system:
				\begin{itemize}
				\item	Allow up to 32 cameras to be connected and usable by the system.
				\end{itemize}
				
				\item	Reliability of the system:
				\begin{itemize}
				\item	Average reliability is required, if the system does crash there should be a watch dog system in place that will restart the system if it goes offline.
				\end{itemize}
				
				\item Security requirements of the system:
				\begin{itemize}
				\item A management portal is to be created to add new users, the system should only be accessed by members of the neighbourhood watch as well as the local police enforcement, these are the only people allowed access to the system.	
				\end{itemize}	
		\end{itemize}
		
		\item Integration and access
			\begin{itemize}
				\item	Interaction of which systems:
				\begin{itemize}				
				\item	IP cameras with mjpg over http or h264 over rstp streams.
				\end{itemize}
				
				\item	Way of interaction (stand alone app/web/mobile):
				\begin{itemize}
				\item	Through the use a mobile application.
				\end{itemize}	
				
			\item Technologies to be used
			\begin{itemize}
				\item	Frameworks:
				\begin{itemize}				
				\item	Opencv is to be used as the base framework.
				\end{itemize}
				
				\item	Programming languages to be used:
				\begin{itemize}
				\item	C++ and Java for android mobile.
				\end{itemize}
				
				\item	Operating Systems
				\begin{itemize}
				\item	Android 2.1 and higher and Linux.
				\end{itemize}
				
				\item	Hardware platforms.
				\begin{itemize}
				\item	x86 based hardware and vivotek 8332 cameras.
				\end{itemize}
				
				
		\end{itemize}
		\end{itemize}
		\end{itemize}
		
		\vspace{0.2in}
		
		\section*{\colorbox{blue}{\makebox[\textwidth-2\fboxsep][l]{\bfseries\color{white} Client commitments}}} \addcontentsline{toc}{section}{Client commitments}
		
		\vspace{0.2in}
		\begin{itemize}
			\item Stuff that will be made available to us from the client: 
			\begin{itemize}
				\item	Resources:
				\begin{itemize}				
				\item	Documents.
				\item	Software.
				\end{itemize}
				
				\item	Hardware
				\begin{itemize}
				\item	Access to the neighbourhood cameras, provided that the software generated for the project is provided free of charge to the community.
				\end{itemize}
				
				\item	Financial support:
				\begin{itemize}
				\item	Workspace at offices with access to cameras will be provided.
				\end{itemize}
				
				\item Update times:
				\begin{itemize}
				\item Weekly updates are required on the progress of the system.	
				\end{itemize}	
				
				\item Financial commitments:
				\begin{itemize}
				\item Small payment to be expected if system solves problem (R5000.00).	
				\end{itemize}	
		\end{itemize}
		\end{itemize}
		
		\vspace{0.2in}
		

\end{document}